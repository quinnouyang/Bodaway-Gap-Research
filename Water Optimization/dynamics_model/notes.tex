\documentclass{article}

\usepackage[colorlinks=true]{hyperref}
\hypersetup{urlcolor=[RGB]{0,0,180}} % A darker blue than the default

\title{Water Dynamics Notes}
\author{Quinn Ouyang}
\date{\today}

\begin{document}

\maketitle

\section{\href{https://doi.org/10.1016/j.agwat.2014.10.008}{Reca, et al.: Optimal pumping scheduling model considering reservoir evaporation}}

\subsection{Previous Work}

\begin{itemize}
    \item Evaporation losses
    \item Cover reservoirs to prevent evaporation; costly for large ones
    \item Optimizing energy costs of pumping schedules, not considering evaporation
\end{itemize}

\subsection{Methodology}

Linear function for surface area $A$ to volume $S$: $A = c_1 \times S + c_2$

Pan evaporation $E_{o}$ tends to be proportionally higher reservoir evaporation .$E_d$
\begin{itemize}
    \item $E_d = K_{r, d} \times E_{o,d}, K_{r, d} >1$ for day $d$
    \item This requires “daily evaporation estimations into an hourly basis,” which is highly variable as it depends more on wind speed than radiation.
\end{itemize}

\subsection{LP Optimization}

\subsubsection{Objective Function}

Optimize total cost $C_e$, given energy price $p$ and consumption $W$ at time period $i$

\begin{itemize}
    \item $C_e = \sum p_i \times W_i$
    \item $W_i = w_i \times V_i$ for unitary consumption $w$ and volume pumped $V$ at time period $i$
    \item Also can include unitary water cost $p_w$
\end{itemize}

\subsubsection{Constraints}

\begin{itemize}
    \item Max pumping capacity: $V_i \leq Q \times N_i, \forall i$ for system discharge rate $Q$ and duration $N$ at period $i$
    \item Max/min storage capacity
    \item $S_i \in [S_m, S_M], \forall i$ where $S_i = S_{i-1} + V_i + R_i - D_i - RE_i$ for demand $D$, rainfall $R$, and evaporation $RE$ at period $i$
    \item Non-negativity
\end{itemize}

\subsection{Case Study}

Instead of pumping water out from reservoirs, this is pumping water from wells into reservoirs, where evaporation will be more of an issue.

\begin{itemize}
    \item Given: area-volume measurements, pumping discharge etc., demand, electricity rates
    \item Measured: evaporation data (via A-class pan and weather station)
    \item Approximated hourly evaporation patterns with nearby data
\end{itemize}

\subsection{Results}

\begin{itemize}
    \item Two models: control (without considering evaporation), above methodology
    \item For control, model anticipated pumping hours by maximizing low cost pumping periods
    \item For methodology, model delayed pumping to minimize stored water pumped into reservoirs
\end{itemize}

\newpage

\section{\href{https://deepcreekanswers.com/deepcreekscience/documents/evaporation/Penpan.pdf}{Linacre: Estimating U.S. Class A Pan Evaporation from Few Climate Data}}

\subsection{Previous Work}

\begin{itemize}
    \item Comparing variable pan measurements with \href{https://en.wikipedia.org/wiki/Penman_equation}{Penman’s equation}
    \item Formula requires temperature, humidity, wind, and irradiance measurements
    \item Lots of pan data, few lake-evaporation measurements to verify Penman’s
\end{itemize}

Purpose is to modify Penman’s equation for a pan to verify its results

\subsection{Methodology}

\subsubsection{Pan}

\begin{itemize}
    \item Wire mesh screen reduces wind and radiation, lowering evaporation
    \item Higher elevations has thinner air, reducing radiation intensity for higher pans
    \item Heat transfer through pan walls, depending on solar irradiance of the ground and sun elevation
    \item Dryer ground increases evaporation
\end{itemize}

\subsubsection{Modified Equation}

Similar to that for a leaf because water loss resistance > heat transfer resistance (?)

\subsubsection{Conclusions}

\begin{itemize}
    \item $E_o = 0.77$
    \item $E_p$ depends on solar irradiance (pan geometry, solar positioning, atmosphere)
    \item Modified equation is more reliable than Penman’s, involving “extraterretrial radiation and rainfall measurements”
\end{itemize}

\newpage

\section{\href{https://ieeexplore.ieee.org/document/8535643}{Ngancha et al.: Optimal Pump Scheduling … Incorporating Evaporation and Seepage Effect}}

\subsection{Previous Work}

Studies at different levels: water storage, pumping, and cost optimization

\subsection{Case Study}

\begin{itemize}
    \item Water treatment system: river water to dams to reservoirs for treatment
    \item Goal: minimize electricity and water costs while satisfying demand and considering evaporation
\end{itemize}

\newpage

\section{\href{https://link.springer.com/article/10.1007/s11269-008-9303-3}{Sivapragasam, et al.: Modeling Evaporation-Seepage Losses …}}

\subsection{Previous Work}

\begin{itemize}
    \item Most research model non-evaporation losses, esp. seepage, as a pre-defined relationship or as negligible
    \item Penman most commonly used for evaporation modeling, with possible some under-estimation
\end{itemize}

\subsection{Methodology}

\begin{itemize}
    \item GP algorithm
          \begin{itemize}
              \item Evaporation: meteorological parameters + surface area (volume / depth)
                    \begin{itemize}
                        \item $E_t = f(h_{t-1}, SA_{t-1}, T_{t_24}, RH_{t-24}, N_{t-24}, V_{t-24})$
                        \item Evaporation(-seepage?) loss $E$, temperature $T$, wind velocity $V$, sunshine hours $N$, relative humidity $RH$, surface area $SA$, and depth $h$ at time $t$
                        \item $t-1$ indicates one fortnight before and $t-24$ indicates the same fortnight one year before
                        \item inflow not explicitly considered because assume depth accounts for that ($h$ also indirectly accounts for $SA$ as they have a linear relationship)
                    \end{itemize}
          \end{itemize}
    \item Seepage: depth (lack of other info, e.g. saturation)
          - Assumption: average climate in a given year is not significantly different from the year before
          - Develop GP equations for both reservoirs in case study, compare to Penman’s
\end{itemize}

\subsection{Conclusions}

\begin{itemize}
    \item GP and Penman’s corr coe. is $0.85$ vs. $0.64$, GP model better for March through July, indicating some losses in those months that are beyond evaporation. i.e. models predict evaporation very similarly, but GP performs better by including non-evaporation modeling
    \item Unsatisfactory predictions from both models because of data frequency and station distance (?)
\end{itemize}

\newpage

\section{Other Sources}

\begin{itemize}
    \item \href{https://core.ac.uk/download/pdf/270268124.pdf}{Small reservoirs depth-area-volume relationships …}
    \item \href{https://www.sciencedirect.com/science/article/pii/S1876610212001968?fr=RR-1&ref=cra_js_challenge}{Establishing Water Surface Area-Storage Capacity Relationship …}
\end{itemize}

\end{document}
